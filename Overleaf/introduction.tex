\chapter{Introduction}
The goal of our project is to create a fully functional web-based point-of-sale system with remote ordering capabilities (via a tablet) for a bar/restaurant that can be easily adopted to a café or any shop or premises that require a till system that can process payments, print receipts, use a barcode scanner as well as being able to generate reports to help managers and owners to make up-to-date, reliable decisions that will benefit the company both in the short and long term.\par
With so many options out there to have food owners and staff alike have to do everything in their power to make sure the customers visit is as pleasant as possible. Having a bad experience at a bar/restaurant can result in a customer never going back. Sometimes these can’t be avoided but sometimes they can. Having a customer waiting a long period looking to pay for their food because of a system fault could be the difference between losing a customer forever or gaining one.\par
The system allows the user to process, with speed and ease, a customers purchases which will keep customers happy and give them a good experience.
The system also offers a customer loyalty system which offers customer discounts and will keep track of customers purchases and purchase dates. It also allows for the storing of customer data such as email address and phone number which can be used to reach out to a customer if they haven’t been in the premises in a while.\par
The chosen languages to accomplish this task were PHP and JavaScript while using MySQL as the database for the project. There is also HTML and CSS for the front-end design purposes.\par
Having both worked in industries where having the right point-of-sale system is critical in the everyday running of a business we decided from an early stage that building an efficient one is what we wanted to do. This project aimed to address any short comings these systems had to better aid the workflow of the user. A good POS system will allow the user to focus on more pressing tasks like customer service rather than having to worry about an unreliable till.\par
Setting out initially we knew that our project had to be worthy of the 15 credits on offer at level 8. This played a deciding factor in deciding to learn a new language to complete the task. After some research we decided we would learn PHP to produce our system with some Javascript too. PHP was chosen as it was an Object-Oriented programming language and it was a language we had not used before and in light of that we thought it would be good to try something new and broaden our own knowledge on the language.\par
Once the languages were sorted out. The task to set out a project scope that we could use and follow begun, some of the base requirements we decided on at the time were that the POS needed a template for php (front end - back end), user authentication (login ect), management accounts for the boss or managers, types of food / drink being "sold" at restaurant ( how to have set up database with both of these ), have items as products (what makes a product? - name , id , price , ect) adding/deleting editing (CRUD), adding a type of loyalty system or adding users to the system for deals on next purchases (CRUD on them too). have a gui till face, able to be placed onto a tablet for the server, have the sales gui (till front end) create reports that can be looked over in the future. Create a payment method , add tax to a sale depending on the type of category sale is tax will vary. add some sort of external hardware eg a ticket printer or barcode scanner or a card reader machine (if can get hands on one). Most of the the previously listed demands are achievable with what we have on hand other than some of the end hardware that we can hopefully acquire as we go along with the project. Some of the major constraints of the project that may take up a lot of time , initially setting up all CRUD architecture and all the routeing in the project , linking database , a method of payment , adding tax that is changed depending on the category of item. getting the sales from the till front end to be displayed on a sales page for management to access and determine the state of the business. We believe that within the given time frame all that was set out to be done should be realistcally possible, in the case of time catching up on us some features that are not essential to the essence of the project such as the extra hardware that is planned to be added at the end of the project (barcode scanner. printer, cardmachine), loyalty system for customers and having multiple different methods of payment can be dropped as they are not essential to the project itself. (NEXT add constraints that we have identified on the way an things that we had to CHANGE)